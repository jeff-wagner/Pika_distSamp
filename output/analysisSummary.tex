% Options for packages loaded elsewhere
\PassOptionsToPackage{unicode}{hyperref}
\PassOptionsToPackage{hyphens}{url}
%
\documentclass[
]{article}
\usepackage{amsmath,amssymb}
\usepackage{lmodern}
\usepackage{iftex}
\ifPDFTeX
  \usepackage[T1]{fontenc}
  \usepackage[utf8]{inputenc}
  \usepackage{textcomp} % provide euro and other symbols
\else % if luatex or xetex
  \usepackage{unicode-math}
  \defaultfontfeatures{Scale=MatchLowercase}
  \defaultfontfeatures[\rmfamily]{Ligatures=TeX,Scale=1}
\fi
% Use upquote if available, for straight quotes in verbatim environments
\IfFileExists{upquote.sty}{\usepackage{upquote}}{}
\IfFileExists{microtype.sty}{% use microtype if available
  \usepackage[]{microtype}
  \UseMicrotypeSet[protrusion]{basicmath} % disable protrusion for tt fonts
}{}
\makeatletter
\@ifundefined{KOMAClassName}{% if non-KOMA class
  \IfFileExists{parskip.sty}{%
    \usepackage{parskip}
  }{% else
    \setlength{\parindent}{0pt}
    \setlength{\parskip}{6pt plus 2pt minus 1pt}}
}{% if KOMA class
  \KOMAoptions{parskip=half}}
\makeatother
\usepackage{xcolor}
\IfFileExists{xurl.sty}{\usepackage{xurl}}{} % add URL line breaks if available
\IfFileExists{bookmark.sty}{\usepackage{bookmark}}{\usepackage{hyperref}}
\hypersetup{
  pdftitle={Collared Pika Distance Sampling Analysis},
  pdfauthor={Jeff Wagner},
  hidelinks,
  pdfcreator={LaTeX via pandoc}}
\urlstyle{same} % disable monospaced font for URLs
\usepackage[margin=1in]{geometry}
\usepackage{longtable,booktabs,array}
\usepackage{calc} % for calculating minipage widths
% Correct order of tables after \paragraph or \subparagraph
\usepackage{etoolbox}
\makeatletter
\patchcmd\longtable{\par}{\if@noskipsec\mbox{}\fi\par}{}{}
\makeatother
% Allow footnotes in longtable head/foot
\IfFileExists{footnotehyper.sty}{\usepackage{footnotehyper}}{\usepackage{footnote}}
\makesavenoteenv{longtable}
\usepackage{graphicx}
\makeatletter
\def\maxwidth{\ifdim\Gin@nat@width>\linewidth\linewidth\else\Gin@nat@width\fi}
\def\maxheight{\ifdim\Gin@nat@height>\textheight\textheight\else\Gin@nat@height\fi}
\makeatother
% Scale images if necessary, so that they will not overflow the page
% margins by default, and it is still possible to overwrite the defaults
% using explicit options in \includegraphics[width, height, ...]{}
\setkeys{Gin}{width=\maxwidth,height=\maxheight,keepaspectratio}
% Set default figure placement to htbp
\makeatletter
\def\fps@figure{htbp}
\makeatother
\setlength{\emergencystretch}{3em} % prevent overfull lines
\providecommand{\tightlist}{%
  \setlength{\itemsep}{0pt}\setlength{\parskip}{0pt}}
\setcounter{secnumdepth}{-\maxdimen} % remove section numbering
\ifLuaTeX
  \usepackage{selnolig}  % disable illegal ligatures
\fi

\title{Collared Pika Distance Sampling Analysis}
\author{Jeff Wagner}
\date{}

\begin{document}
\maketitle

This is a breakdown of an analysis of collared pika \emph{(Ochotona
collaris)} data collected by the Alaska Center for Conservation Science
in 2018-2019. The data was collected following a spatially-explicit
distance sampling approach (Royle et al.~2004). Analysis was conducted
in Program R (version 4.2.0) with package \emph{unmarked} (version
1.2.5) and predictors were compiled via an ArcGIS Pro Python
installation (version 3.6+).

\hypertarget{top-model-set}{%
\subsubsection{\texorpdfstring{\textbf{Top Model
Set:}}{Top Model Set:}}\label{top-model-set}}

\begin{longtable}[]{@{}
  >{\raggedright\arraybackslash}p{(\columnwidth - 16\tabcolsep) * \real{0.0270}}
  >{\raggedright\arraybackslash}p{(\columnwidth - 16\tabcolsep) * \real{0.2072}}
  >{\raggedleft\arraybackslash}p{(\columnwidth - 16\tabcolsep) * \real{0.0270}}
  >{\raggedleft\arraybackslash}p{(\columnwidth - 16\tabcolsep) * \real{0.0811}}
  >{\raggedleft\arraybackslash}p{(\columnwidth - 16\tabcolsep) * \real{0.0991}}
  >{\raggedleft\arraybackslash}p{(\columnwidth - 16\tabcolsep) * \real{0.1532}}
  >{\raggedleft\arraybackslash}p{(\columnwidth - 16\tabcolsep) * \real{0.1081}}
  >{\raggedleft\arraybackslash}p{(\columnwidth - 16\tabcolsep) * \real{0.1351}}
  >{\raggedleft\arraybackslash}p{(\columnwidth - 16\tabcolsep) * \real{0.1622}}@{}}
\caption{AICc model selection table}\tabularnewline
\toprule
\begin{minipage}[b]{\linewidth}\raggedright
\end{minipage} & \begin{minipage}[b]{\linewidth}\raggedright
Model
\end{minipage} & \begin{minipage}[b]{\linewidth}\raggedleft
K
\end{minipage} & \begin{minipage}[b]{\linewidth}\raggedleft
AICc
\end{minipage} & \begin{minipage}[b]{\linewidth}\raggedleft
Delta AICc
\end{minipage} & \begin{minipage}[b]{\linewidth}\raggedleft
Model Likelihood
\end{minipage} & \begin{minipage}[b]{\linewidth}\raggedleft
AICc Weight
\end{minipage} & \begin{minipage}[b]{\linewidth}\raggedleft
Log Likelihood
\end{minipage} & \begin{minipage}[b]{\linewidth}\raggedleft
Cumulative Weight
\end{minipage} \\
\midrule
\endfirsthead
\toprule
\begin{minipage}[b]{\linewidth}\raggedright
\end{minipage} & \begin{minipage}[b]{\linewidth}\raggedright
Model
\end{minipage} & \begin{minipage}[b]{\linewidth}\raggedleft
K
\end{minipage} & \begin{minipage}[b]{\linewidth}\raggedleft
AICc
\end{minipage} & \begin{minipage}[b]{\linewidth}\raggedleft
Delta AICc
\end{minipage} & \begin{minipage}[b]{\linewidth}\raggedleft
Model Likelihood
\end{minipage} & \begin{minipage}[b]{\linewidth}\raggedleft
AICc Weight
\end{minipage} & \begin{minipage}[b]{\linewidth}\raggedleft
Log Likelihood
\end{minipage} & \begin{minipage}[b]{\linewidth}\raggedleft
Cumulative Weight
\end{minipage} \\
\midrule
\endhead
6 & climate \& productivity & 8 & 717.5533 & 0.000000 & 1.0000000 &
0.5530377 & -350.1221 & 0.5530377 \\
2 & climate & 7 & 718.1131 & 0.559722 & 0.7558888 & 0.4180350 &
-351.5520 & 0.9710726 \\
4 & climate \& topography & 10 & 724.1778 & 6.624476 & 0.0364345 &
0.0201497 & -351.0704 & 0.9912223 \\
5 & productivity & 6 & 726.6358 & 9.082469 & 0.0106602 & 0.0058955 &
-356.9429 & 0.9971178 \\
1 & null & 4 & 728.3336 & 10.780211 & 0.0045615 & 0.0025227 & -359.9913
& 0.9996405 \\
3 & topography & 7 & 732.2302 & 14.676900 & 0.0006501 & 0.0003595 &
-358.6106 & 1.0000000 \\
\bottomrule
\end{longtable}

\hypertarget{climate-productivity-model}{%
\subsubsection{\texorpdfstring{\emph{Climate \& Productivity
Model}}{Climate \& Productivity Model}}\label{climate-productivity-model}}

\hypertarget{summary-and-parameter-estimates}{%
\subparagraph{Summary and Parameter
Estimates}\label{summary-and-parameter-estimates}}

\begin{verbatim}
## 
## Call:
## distsamp(formula = ~scale(search.speed) ~ scale(precip) + scale(summerWarmth) + 
##     scale(logs) + scale(wetness), data = umf, keyfun = "hazard", 
##     output = "density", unitsOut = "kmsq")
## 
## Density (log-scale):
##                     Estimate    SE      z  P(>|z|)
## (Intercept)           3.3709 0.187 18.002 1.87e-72
## scale(precip)         0.2546 0.132  1.935 5.30e-02
## scale(summerWarmth)  -0.1125 0.131 -0.862 3.89e-01
## scale(logs)          -0.3369 0.144 -2.341 1.92e-02
## scale(wetness)        0.0572 0.092  0.622 5.34e-01
## 
## Detection (log-scale):
##                     Estimate    SE     z  P(>|z|)
## (Intercept)            1.807 0.302  5.99 2.13e-09
## scale(search.speed)   -0.919 0.139 -6.60 4.08e-11
## 
## Hazard-rate(scale) (log-scale):
##  Estimate    SE    z  P(>|z|)
##     0.645 0.144 4.47 7.67e-06
## 
## AIC: 716.2442 
## Number of sites: 119
## optim convergence code: 0
## optim iterations: 66 
## Bootstrap iterations: 0 
## 
## Survey design: line-transect
## Detection function: hazard
## UnitsIn: m
## UnitsOut: kmsq
\end{verbatim}

\begin{verbatim}
##                                 0.025       0.975
## lam(Int)                  3.003864741  3.73785253
## lam(scale(precip))       -0.003322409  0.51251299
## lam(scale(summerWarmth)) -0.368336709  0.14337161
## lam(scale(logs))         -0.618936678 -0.05488722
## lam(scale(wetness))      -0.123144545  0.23755487
\end{verbatim}

\hypertarget{plot}{%
\subparagraph{Plot}\label{plot}}

\includegraphics{analysisSummary_files/figure-latex/unnamed-chunk-4-1.pdf}

\hypertarget{climate-model}{%
\subsubsection{\texorpdfstring{\emph{Climate
Model}}{Climate Model}}\label{climate-model}}

\hypertarget{summary-and-parameter-estimates-1}{%
\subparagraph{Summary and Parameter
Estimates}\label{summary-and-parameter-estimates-1}}

\begin{verbatim}
## 
## Call:
## distsamp(formula = ~scale(search.speed) ~ scale(precip) + scale(summerWarmth) + 
##     scale(januaryMinTemp), data = umf, keyfun = "hazard", output = "density", 
##     unitsOut = "kmsq")
## 
## Density (log-scale):
##                       Estimate     SE     z  P(>|z|)
## (Intercept)              3.375 0.1808 18.66 9.56e-78
## scale(precip)            0.147 0.1042  1.41 1.59e-01
## scale(summerWarmth)     -0.314 0.1165 -2.69 7.11e-03
## scale(januaryMinTemp)   -0.203 0.0972 -2.09 3.68e-02
## 
## Detection (log-scale):
##                     Estimate    SE     z  P(>|z|)
## (Intercept)            1.868 0.285  6.55 5.91e-11
## scale(search.speed)   -0.875 0.131 -6.67 2.61e-11
## 
## Hazard-rate(scale) (log-scale):
##  Estimate    SE    z  P(>|z|)
##     0.668 0.143 4.68 2.93e-06
## 
## AIC: 717.1041 
## Number of sites: 119
## optim convergence code: 0
## optim iterations: 43 
## Bootstrap iterations: 0 
## 
## Survey design: line-transect
## Detection function: hazard
## UnitsIn: m
## UnitsOut: kmsq
\end{verbatim}

\begin{verbatim}
##                                  0.025       0.975
## lam(Int)                    3.02045512  3.72922671
## lam(scale(precip))         -0.05730811  0.35119781
## lam(scale(summerWarmth))   -0.54211845 -0.08528593
## lam(scale(januaryMinTemp)) -0.39341201 -0.01239293
\end{verbatim}

\hypertarget{plot-1}{%
\subparagraph{Plot}\label{plot-1}}

\includegraphics{analysisSummary_files/figure-latex/unnamed-chunk-6-1.pdf}

\end{document}
